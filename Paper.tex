%% LyX 2.1.3 created this file.  For more info, see http://www.lyx.org/.
%% Do not edit unless you really know what you are doing.
\documentclass[11pt]{article}
\usepackage{amsmath}
\usepackage{amssymb}
\usepackage{fontspec}
\usepackage[a4paper]{geometry}
\geometry{verbose,tmargin=2cm,bmargin=1.5cm,lmargin=2cm,rmargin=2cm}
\usepackage{fancyhdr}
\pagestyle{fancy}
\usepackage{color}
\usepackage{array}
\usepackage{prettyref}
\usepackage{multirow}
\usepackage{graphicx}
\usepackage[unicode=true,pdfusetitle,
 bookmarks=true,bookmarksnumbered=false,bookmarksopen=true,bookmarksopenlevel=1,
 breaklinks=true,pdfborder={0 0 0},backref=false,colorlinks=true]
 {hyperref}
\hypersetup{
 linkcolor=blue, urlcolor=red, citecolor=blue}

\makeatletter

%%%%%%%%%%%%%%%%%%%%%%%%%%%%%% LyX specific LaTeX commands.
%% Because html converters don't know tabularnewline
\providecommand{\tabularnewline}{\\}

%%%%%%%%%%%%%%%%%%%%%%%%%%%%%% Textclass specific LaTeX commands.
\newcommand{\lyxaddress}[1]{
\par {\raggedright #1
\vspace{1.4em}
\noindent\par}
}

%%%%%%%%%%%%%%%%%%%%%%%%%%%%%% User specified LaTeX commands.
%%% Packages %%%
\usepackage{graphicx}
\usepackage{marginnote}

%%% Bibliography %%%
\usepackage[style=annotated,citestyle=numeric,terseinits=true,minbibnames=10,maxbibnames=10,sorting=none,sortcites=true,natbib=true,backend=biber]{biblatex}
\addbibresource{references.bib}
\usepackage{enumitem}
% Margin notes on left
\usepackage{marginnote}
\reversemarginpar
\newrefformat{fig}{Figure~\ref{#1}}
\newrefformat{tab}{Table~\ref{#1}}
\date{}

\makeatother

\usepackage{xunicode}
\begin{document}

\title{Pheromone processing in \emph{Drosophila}}


\author{Johannes Kohl\textsuperscript{1,2,3}, Paavo Huoviala\textsuperscript{1}
and Gregory S X E Jefferis\textsuperscript{1,3}}

\maketitle

\lyxaddress{\textsuperscript{1} Division of Neurobiology, MRC Laboratory of
Molecular Biology, Cambridge, CB2 0QH, UK}


\lyxaddress{\textsuperscript{2 }Department of Molecular and Cellular Biology,
Harvard University, Cambridge, MA, USA}


\lyxaddress{\textsuperscript{3 }Corresponding authors: Kohl, Johannes (jkohl@fas.harvard.edu),
Jefferis, Gregory S X E (jefferis@mrc-lmb.cam.ac.uk)}




\section*{Highlights }


\begin{itemize}
\item Distinct sensory systems for volatile and contact pheromone detection
\item Identified \emph{Drosophila} pheromones and their receptors
\item Central processing of pheromone signals
\item Sex-specific rerouting of pheromone signals 
\end{itemize}

\section*{Graphical abstract}




\section*{Abstract }

Understanding how sensory stimuli are processed in the brain to instruct
appropriate behavior is a fundamental question in neuroscience. \emph{Drosophila}
has become a powerful model system to address this problem. Recent
advances in characterizing the circuits underlying pheromone processing
have put the field in a position to follow the transformation of these
chemical signals all the way from the sensory periphery to decision
making and motor output. Here we describe the latest advances, outline
emerging principles of pheromone processing and discuss future questions. 


\section*{Introduction}

Pheromones are powerful chemosensory stimuli, released into the environment
by individuals with the ultimate goal of altering the behavior or
physiology of others members of their species. Since the first description
of pheromones, tremendous advances have been made in our understanding
of how these chemicals are detected by sensory systems. However, with
one notable exception (see below) our understanding of how pheromones
are processed in the brain is still relatively limited. \emph{Drosophila}
is an attractive model system for studying the logic of pheromone
perception since (1) pheromone-receptor pairs have been identified,
(2) most of its \textasciitilde{}100,000 neurons are accessible to
genetic manipulations and (3) pheromones modulate several, quantifiable
behaviors. Since sensory pheromone detection in \emph{Drosophila}
has been extensively reviewed elsewhere (e.g. see \citep{Sengupta2014})
we will only briefly outline its principles before discussing recent
advances in understanding pheromone processing deeper in the fly brain
(i.e. perception). Finally, we will address sex differences in pheromone
processing and outline future challenges. Although putative larval
pheromones have recently been described \citep{Farine:2014ve,Mast:2014aa},
this review will focus on pheromone processing in the adult fly.


\subsection*{Pheromone detection in \emph{Drosophila}}

Pheromone-sensitive cells in \emph{Drosophila} are incorporated into
the chemosensory systems of smell and taste. Flies, like most insects,
have distinct anatomical subsystems for detection of volatile (long-range)
and non-volatile (contact) chemicals: olfactory receptor neurons (ORNs)
on head appendages and gustatory receptor neurons (GRNs) on various
body parts \citep{Stocker:1994fu,Montell:2009fk}. 

Odors are detected by odorant receptor (OR) or ionotropic receptor
(IR) -expressing ORNs housed in sensory bristles (sensilla) on the
third antennal segment and maxillary palps \citep{Vosshall:2007nx,Benton:2009ly}.
On each antenna, \textasciitilde{}45 classes of ORNs \citep{Vosshall:2007nx}
are arranged as clusters of 1--4 neurons in four morphologically distinct
classes of sensilla (basiconic, coeloconic, intermediate and trichoid).
Intriguingly, only trichoid ORNs seem to be responsive to fly odors
\citep{Goes-van-Naters:2007dq}. Axons from 20--50 ORNs (i.e. first-order
olfactory neurons) expressing the same OR / IR converge on each of
\textasciitilde{}50 glomeruli in the antennal lobe (AL), the first
relay station for olfactory information. There they form synapses
with 1--7 excitatory or inhibitory projection neurons (PNs, second-order
olfatory neurons) and a complex network of excitatory and inhibitory
local neurons (LNs) \citep{Olsen:2007dz,Shang:2007vn,Chou:2010uq}.
PNs then relay the olfactory information to two distinct higher brain
centers, the mushroom body (MB, required for olfactory learning) and
the lateral horn (LH) \citep{Jefferis:2007zr}. 

In contrast, non-volatile chemicals are detected via taste sensilla
on the labellum, pharynx, legs, wing margins and ovipositor \citep{Stocker:1994fu,Montell:2009fk,Liman:2014bs}.
Most sensilla contain four GRNs that express one of either \textasciitilde{}60
gustatory receptor (GR) or \textasciitilde{}35 IR genes conferring
sensitivity to sweet, bitter, salt, water, fatty acids or carbonation
\citep{Amrein:2005uq,Scott:2005kx,Thistle:2012fu,Koh:2014cr}. Several
of these GRs and IRs are candidate pheromone receptors; all are housed
in labellar or tarsal taste sensilla. GRNs project to the gnathal
ganglion (GNG, formerly the subesophageal ganglion), the putative
first-order processing center for gustatory information and to specific
thoracic ganglia in the ventral nerve cord. In the GNG, their projections
are reported to segregate by gustatory organ and taste category, but
not by receptor, as is the case for ORNs \citep{Wang:2004qo}. 


\subsection*{\emph{Drosophila} pheromonal stimuli and their receptors}



While the tuning of most \emph{Drosophila} ORs, IRs and many GRs \citep{Hallem:2006bs,Montell:2009fk}
has been investigated using a wide range of stimuli, barely any pheromone
receptor-ligand pairs have been identified. A handful of trichoid
ORs were found to respond to volatile compounds in fly extracts \citep{Goes-van-Naters:2007dq},
but the chemical identity of the ligands remains unknown, with one
exception: 11-\emph{cis}-vaccenyl acetate (cVA), produced in the male
ejaculatory bulb and transferred to females during copulation, is
detected by two narrowly tuned ORs in both sexes, OR67d and OR65a
\citep{Ha:2006qa,Kurtovic:2007fu,Goes-van-Naters:2007dq}. Similarly,
only a small number of pheromone-responsive GRs has been identified.
They are likely activated by cuticular hydrocarbons (CHCs) \citep{Ferveur:2005ly},
long-chain fatty acids which are produced by oenocytes, specialized
cells located on the inner surface of the abdominal cuticle \citep{Billeter:2009qf}.
Some of these CHCs have been shown to be volatile \citep{Farine:2012bh},
suggesting that they might be detected through olfaction as well as
taste. CHCs are an essential sensory component for courtship \citep{Billeter:2009qf}
and courtship-inhibiting CHCs are present on both males and females.
At least three of these courtship-inhibiting CHCs, 7-T, 9-T and 11-P
\citep{Ferveur:2005ly}, which are secreted by conspecific males or
flies of other species, are detected by Gr32a \citep{Miyamoto:2008kl,Fan:2013fv}
(and probably Gr33a \citep{Moon:2009qa}). Intriguingly, both Gr32a
and Gr33a are also required for detection of bitter-tasting compounds
\citep{Moon:2009qa}, suggesting that reproductive dead ends have
aversive valence \citep{Lacaille:2007ij}. Another receptor, Gr68a,
which is expressed in chemosensory neurons of \textasciitilde{}20
male-specific gustatory bristles in the forelegs, was proposed to
detect female contact pheromones \citep{Bray:2003fk}, although this
finding has been called into question \citep{Ejima:2007bs,Toda:2012ff}.
Of these putative female CHC aphrodisiacs, two have been identified,
7,11-HD and 7,11-ND \citep{Ferveur:2005ly}, but their receptors remain
unknown. Two potential candidates are IR52c and IR52d, which are co-expressed
in male-specific foreleg sensilla and activated by female stimuli
\citep{Koh:2014cr}. Surprisingly, 7,11-HD acts both as an attractant
for conspecific males and as a powerful anti-aphrodisiac for males
of other \emph{Drosophilids} \citep{Billeter:2009qf}, illustrating
that contact pheromones can serve as species barriers. A separate
population of GRNs, expressing \emph{ppk25}, \emph{ppk23} and \emph{ppk29},
members of the pickpocket/degenerin-epithelial (Ppk/DEG-ENaC) family
of sodium channels, was found to be necessary for detection of contact
pheromones involved in male courtship behavior \citep{Lin:2005pi,Toda:2012ff,Thistle:2012fu,Starostina:2012mi,Liu:2012ys,Vijayan:2014vn}.
\emph{ppk23} marks a subset of paired neurons in male-specific chemosensory
leg bristles. One of these neurons responds to male pheromones (7-T,
cVA), whereas the other one, characterized by \emph{ppk25} co-expression,
responds to female pheromones (7,11-ND, 7,11-HD) \citep{Lin:2005pi,Thistle:2012fu,Starostina:2012mi,Toda:2012ff,Vijayan:2014vn}.
Interestingly, these \emph{ppk} neurons express neither Gr32a nor
Gr68a \citep{Thistle:2012fu}. A picture emerges therefore in which
parallel, but functionally complementary systems exist for contact
pheromone detection in \emph{Drosophila}. 


\subsection*{Central circuits for pheromone perception }

Olfactory information carried by PNs converges, apparently randomly,
onto third-order neurons in the MB (Kenyon Cells) \citep{Caron:2013vn}
which integrate inputs from different glomeruli linearly or sublinearly
\citep{Gruntman:2013vn}. This probabilistic wiring is thought to
reflect the MB's role in learning and memory. In contrast, third-order
lateral horn neurons (LHNs) exhibit stereotyped connectivity and odor
responses \citep{Liang:2013nx,Kohl:2013kx,Fisek:2014uq}, in accordance
with the LH's hypothesized role in mediating innate olfactory behaviors.
Since most ORNs are broadly tuned, odorant identity is presumably
encoded by combinatorial activity of ORN -> PN ensembles. While this
distributive model is appropriate for general odor coding (but see
Mansourian \& Stensmyr {[}this issue{]} for a discussion of combinatorial
coding under natural odor concentrations), the only identified olfactory
pheromone, cVA, seems be initially processed by a \emph{labeled line}
instead, i.e. a neural circuit dedicated to a specific sensory stimulus. 

Following the cVA signal into the brain has given us unique insights
into the logic of pheromone perception. The tuning of first-, second-
and some third-order olfactory neurons to cVA is extremely narrow
\citep{Schlief:2007fk,Datta:2008dz,Ruta:2010ys,Kohl:2013kx}\emph{
}and\emph{ }all components of this circuit express male-specific transcripts
of \emph{fruitless} \citep{Kurtovic:2007fu,Datta:2008dz,Ruta:2010ys,Cachero:2010vn},
a transcription factor thought to instruct the formation of neural
circuits underlying male courtship behavior \citep{Stockinger:2005tg}.
Several features distinguish pheromone processing from that of general
odors: cVA (as well as volatile fly extracts) elicits relatively weak
responses in ORNs \citep{Goes-van-Naters:2007dq}, a potential consequence
of its relatively low volatility. However, these weak responses are
strongly amplified in postsynaptic DA1 PNs \citep{Schlief:2007fk}
(of which there are 6--8, the highest reported for any glomerulus).
In addition, \citet{Chou:2010uq} found that the innervation density
of some LNs was significantly lower in the pheromone-processing glomeruli
DA1, DL3 and VA1d, and that these ``pheromone-avoiding'' LNs fired
a significantly higher percentage of their spikes during the first
100 ms of the odor response as compared to all other LNs. Apart from
cholinergic PNs mediating excitatory, feedforward transmission, five
classes of GABAergic, inhibitory PNs (iPNs) have been identified,
three of which target the \emph{fru+} glomeruli DA1, VL2a and VA1lm
\citep{Jefferis:2007zr}. While these iPNs potently inhibit LHN responses
to food odors, responses to phero- and kairomonal stimuli (via DA1
and VL2a) are iPN-independent \citep{Liang:2013nx}. Interestingly,
\citet{Hong:2015uq} recently found that ORNs vary dramatically in
their sensitivity to inhibitory LN activation in the AL and that DA1
is particularly sensitive to this presynaptic inhibition \citep{Hong:2015uq}.
Therefore, different types of inhibition within pheromone-processing
circuits might operate according to different rules. 

Taken together, these circuit characteristics (signal amplification,
numerical robustness, exemption from some types of inhibition) might
ensure high-fidelity transmission of pheromone information even in
the presence of other olfactory stimuli. However, since cVA is the
only identified volatile pheromone in \emph{Drosophila}, it remains
unclear whether these findings can be generalized.

Even less is known about central processing of contact pheromones
(or any gustatory stimuli) in flies. As mentioned, GRNs projections
in the GNG roughly segregate by organ location and taste category
\citep{Wang:2004qo}. Indeed even though individual GRNs can detect
multiple taste qualities, these tend to have the same valence \citep{Liman:2014bs}.
Therefore, taste in \emph{Drosophila} seems to be a ``valence labeled
line'' modality \citep{Liman:2014bs}. \citet{Kain:2015fk} recently
identified sweet-responsive gustatory PNs downstream of labellar \emph{Gr5a}
GRNs and traced their axonal projections to the antennal mechanosensory
and motor center (AMMC) \citep{Kain:2015fk}. However, since this
represents the only known class of of second-order gustatory neurons
so far, it is unclear whether the 'valence labeled line' organization
is preserved in higher brain centers. 


\subsection*{Integration of pheromone signals}

Why do flies use such a wide variety of receptors and structures to
detect pheromone stimuli that may have similar meaning? Male courtship,
a complex sequence of behaviors (\prettyref{fig:courtship}a), is
uniquely suited to explore this question, since detection of both
volatile and contact pheromones has been shown to coordinate this
elaborate ritual, together with visual, auditory and mechanosensory
cues \citep{Pavlou:2013uq}.

Chemical cues vary widely in their volatility and the same cue may
be detected by different sensory modalities at different ranges. Thus
pheromones may be used to signal not just valence, but also reflect
target distance, potentially acting sequentially to trigger the appropriate
steps of courtship behavior. Initially, cVA in conjunction with food
odors could act as a long- or medium-range aggregation pheromone to
attract flies of both sexes (\prettyref{fig:courtship}a) \citep{Bartelt:1985bh,Schlief:2007fk,Grosjean:2011fu}
to food substrates suitable for mating. At closer range, and while
interacting with other flies in a group, rapidly changing concentrations
of cVA and other sex-specific volatile pheromones (as well as visual
cues) could inform the selection of appropriate courtship targets
(i.e. virgin females) worthy of following. Subsequently, detection
of CHCs via direct contact (tapping) might represent a check point
for deciding between continuation of courtship, i.e. singing, licking
(female of correct species), fighting (male of correct species) or
withdrawal (incorrect species). Finally, if a decision to continue
with courtship is made, detection of secreted female pheromones via
labellar receptors (licking) and auditory cues could gate the final
step of the ritual, copulation. Although simplistic, this working
model of the behavioral algorithm governing courtship behavior makes
testable predictions: 

(a) food and pheromone odors are integrated in the brain and (b) neural
representations of volatile and contact pheromones interact, with
perception of long-range signals possibly gating subsequent perception
of short-range signals, either by modifying approach behavior or directly
at the neural circuit level. 

Some evidence exists in support of the first prediction, i.e. interactions
between the neural representations of volatile food and pheromone
stimuli. While PNs tuned to general odors target dorsal regions of
the LH, PNs downstream of pheromone-responsive trichoid ORNs (DA1,
VL2a, VA1lm, DL3) project to the anterioventral LH \citep{Jefferis:2007zr,Grosjean:2011fu}.
This observation suggests that a spatially segregated representation
of pheromones persists from the AL to the LH \citep{Jefferis:2007zr},
but a few non-pheromonal chemicals that mediate innate attraction
or aversion are also detected by second-order neurons that project
to the vLH: farnesol (Or83c/DC3), geosmin (Or56a, DA2), PAA/PA (Ir84a,
VL2a) (\prettyref{tab:pheromones-receptors}) and valencene (Or19a,
DC1). Since processing of these odors stimuli shares several characteristics
with pheromone processing (innate valence, narrowly tuned PNs with
vLH projections), they can be described as \emph{kairomones}, i.e.
chemicals emitted by an organism, which attract exploiters of another
species. Indeed, co-processing of pheromonal (e.g. cVA) and such kairomonal
(e.g. PAA) stimuli has been suggested to serve the coordination of
feeding and oviposition site selection with reproductive behaviors
\citep{Grosjean:2011fu}. This is supported by the observations that
cVA acts as an aggregation pheromone only in presence of attractive
food odors \citep{Bartelt:1985bh,Schlief:2007fk} and that the food
odor PAA is only attractive when presented with fly odors \citep{Grosjean:2011fu}.
The site of convergence of pheromonal and kairomonal signals with
similar biological value might be found in third-order LHNs with dendrites
in the vLH that integrate signals from both channels. As a consequence
of such interactions, the labeled line model is probably only valid
for the initial stages of pheromone processing. 

The second prediction is that crosstalk between olfactory and contact
pheromone signals exists. This notion is backed, amongst others, by
the observations that Gr32a is required for the aggression-promoting
effect of cVA \citep{Wang:2011hc}, that 7,11-HD mitigates the deterrent
effects of cVA \citep{Billeter:2009qf} and that increased courtship
caused by male CHC ablation is suppressed by a mutation in Or47b \citep{Wang:2011hc}.
Furthermore, since cVA is detected by both olfactory and contact chemosensory
neurons \citep{Ha:2006qa,Kurtovic:2007fu,Goes-van-Naters:2007dq,Thistle:2012fu}
(\prettyref{tab:pheromones-receptors}) -- interactions between both
pheromone processing circuits are likely. This interaction probably
does not occur between GRNs and central olfactory neurons, since neither
DA1 PNs nor identified cVA-responsive LHNs project to the GNG. However,
Gr32a neurons are anatomically poised to relay pheromone information
to sexually dimorphic \emph{fru+} neurons in the GNG \citep{Koganezawa:2010tg}
and the ventrolateral protocerebrum \citep{Miyamoto:2008kl}, which
in turn could connect to the \emph{fru+} cVA olfactory circuit. 




\subsection*{Sexually dimorphic and state-dependent pheromone perception}

How can a single pheromone, cVA, elicit aggression and male-male repulsion
in males \citep{Kurtovic:2007fu,Wang:2010kl,Liu:2011dq}, increase
receptivity in females \citep{Kurtovic:2007fu} and act as aggregation
signal in both sexes \citep{Bartelt:1985bh} (\prettyref{fig:Questions})?
Numerous differences exist between male and female fly brains, most
controlled by \emph{fruitless} \citep{Cachero:2010vn} and \emph{doublesex}
\citep{Rideout:2010dq}. Many of these differences are associated
with sensory processing pathways \citep{Cachero:2010vn}. Males have
a higher number of trichoid sensilla \citep{Stocker:1994fu} and three
enlarged glomeruli, DA1, VA1v and VL2a, all of which express \emph{fru+}
\citep{Stockinger:2005tg}. Deeper in the brain, DA1 PN axons show
a male-specific ventral arborization \citep{Datta:2008dz}, but it
remains unclear if this has functional consequences. Surprisingly,
cVA responses in first- and second-order olfactory neurons are non-dimorphic
\citep{Kurtovic:2007fu,Goes-van-Naters:2007dq,Datta:2008dz}, but
the dendrites of two groups of \emph{fru+} third-order olfactory neurons
sex-specifically overlap with DA1 PN axon terminals, thereby forming
a bidirectional circuit switch that reroutes the cVA signal between
males and females \citep{Kohl:2013kx}. Intriguingly, the position
of this switch is controlled by \emph{fruitless} acting cell-autonomously
in third-order neurons \citep{Kohl:2013kx}. Since only one candidate
fourth-order neuron has been identified so far, the male specific
\emph{fru+} DN1 neuron which projects to the VNC \citep{Ruta:2010ys},
it will now be critical to study how cVA processing after this circuit
bifurcation instructs distinct behaviors. The circuit switch provides
a simple conceptual explanation of how the same sensory stimulus can
elicit sex-specific behaviors, but it does not address (1) how cVA
can act as both repellent and aggregation signal in males and (2)
why prolonged cVA exposure leads to response inhibition in both sexes
\citep{Ejima:2007bs,Liu:2011dq}. As mentioned above, cVA is processed
by two parallel channels \citep{Ha:2006qa,Kurtovic:2007fu,Goes-van-Naters:2007dq}.
While elevated aggression and male-male repulsion are presumably mediated
via acute perception of cVA through the high-affinity Or67d->DA1 channel
\citep{Kurtovic:2007fu,Wang:2010kl}, chronic cVA exposure might recruit
the low-affinity Or65a->DL3 channel and suppress aggression by lateral
inhibition of DA1 in the AL \citep{Ejima:2007bs,Liu:2011dq}. Activation
of Or65a->DL3 (neither of which express \emph{fru}) might therefore
mediate the non-dimorphic aggregation effect of cVA. Lateral inhibition
of DA1 by DL3 might also to be responsible for the abolished attraction
of females to cVA after prolonged exposure \citep{Lebreton:2014cr},
but this interglomerular interaction remains to be directly tested.
Also, in the absence of functional recordings from DL3 PNs or their
postsynaptic neurons, it remains unclear whether additional interactions
exist between both cVA processing channels. Since DL3 PNs project
to the vLH, there are likely to be third-order neurons postsynaptic
to both DA1 and DL3 PNs. 

Sex differences in neuron number and morphology have also been identified
in contact chemoreceptive structures (\prettyref{tab:pheromones-receptors})
\citep{Bray:2003fk,Thistle:2012fu} -- and these dimorphisms are also
controlled by either\emph{ fruitless} or \emph{doublesex}. 

Since the \emph{fru} isoforms sculpting these circuits are male-specific,
female circuits remain drastically understudied. Since female-enlarged
brain regions have been reported \citep{Cachero:2010vn}, it will
be valuable to search for female-specific or -enriched transcripts
which could provide a genetic entry point for studying female fly
behaviors.

Finally, recent work from \citet{Keleman:2012dz} has illustrated
that social experience can modulate hard-wired, pheromone-driven behaviors.
Females decrease their receptivity after mating, one of several profound
physiological changes mediated by transfer of sex peptide (SP) during
copulation \citep{Yapici:2008ve}. Since males also transfer cVA onto
females during mating, this signal allows them to selectively court
appropriate (i.e. receptive) targets. \citet{Keleman:2012dz} observed
that courtship rejection drastically enhances the behavioral sensitivity
of males to cVA and identified a group of \emph{fru+} dopaminergic
neurons that might convey a rejection signal to the MB, thereby producing
lasting changes in cVA processing \citep{Keleman:2012dz}.


\subsection*{Conclusions and future directions}

In recent years, substantial progress has been made in understanding
\emph{Drosophila} pheromone processing, but several fundamental questions
remain (\prettyref{fig:Questions}). At the sensory periphery, the
most obvious frontier is the identification of monomolecular pheromone
\textbf{\emph{ligands}} and their \textbf{\emph{receptors}}. Although
several ORN and GRN populations were identified that are responsive
to complex fly odors and cuticular extracts, respectively, the active
compounds remain unknown. Importantly, the CHCs discovered so far
represent only a small fraction of the complete cuticular profile
of \emph{Drosophila} \citep{Ferveur:2005ly,Everaerts:2010zr}. Only
recently, a novel acetylated hydrocarbon, CH503, was identified which,
like cVA, is transferred from males to females during mating and subsequently
acts as a long-lived inhibitor of male courtship \citep{Yew:2009fv}.
Approaches such as gas-chromatography-linked single-sensillum recordings
\citep{Stensmyr:2012cr} could be used to address these questions.

Deeper in the brain, most of our current knowledge about \textbf{\emph{pheromone
perception }}stems from following the cVA signal along a circuit hardwired
by \emph{fruitless}. It remains open whether additional labeled lines
exist for other volatile or contact pheromones (\prettyref{fig:Questions}).
Identification of corresponding genetic \textbf{\emph{markers}} would
allow us to investigate whether the circuit architecture underlying
cVA processing can be generalized to other pheromones. Our understanding
of contact pheromone processing faces an additional hurdle: we are
only beginning to uncover the identity of second-order gustatory neurons.
This is also the limiting factor for understanding how gustatory and
olfactory (pheromone) signals might be integrated. However, with the
availability of brain-wide neuron atlases and an ever-expanding palette
of driver lines, these questions can certainly be addressed in the
near future. We are therefore confident that \emph{Drosophila} will
remain an invaluable model system for studying the neural circuit
logic of pheromone perception and, more generally, chemosensory processing.


\section*{}


\section*{Conflict of interest statement}

Nothing declared.


\section*{Acknowledgements}

We thank J-C Billeter, S. Frechter, D. Mersch, M. Stensmyr and T.
Wyatt for helpful comments on the manuscript, L. Hillier for help
in graphical design of figure 1, and M. Sokolowski for allowing adapting
parts of figure 1 from her review. We apologize to colleagues whose
work could not be cited here due to space constraints. This work was
supported by a European Research Council Starting Investigator grant,
the European Molecular Biology Organization (EMBO) Young Investigator
Programme, and Medical Research Council (MRC) Grant MC-U105188491
(to G.S.X.E.J.); and MRC Laboratory of Molecular Biology Graduate
Scholarships (to J.K. and P.H.).


\section*{References and recommended reading}

On {[}2{]}:By using a combination of behavioral experiments and gas
chromatography-mass spectrometry this interesting study identifies
larval and adult chemicals that possibly act as pheromonal cues guiding
aggregation and dispersal behavior in the larvae of two Drosophila
species.

On {[}56{]}: This study describes the neural and molecular mechanisms
underlying courtship learning. The authors show that a specific class
of dopaminergic neurons together with the mushroom body are responsible
for the experience-dependent decrease in courtship towards non-virgin
female targets.

On {[}62{]}: By using genetic mosaics the authors manage to localize
the neural basis of male courtship behavior to two groups of fruitless-expressing
neurons, which are also currently prime candidates for putative integration
sites for multimodal pheromonal information. They also develop a novel
method of studying male courtship behaviour by using tethered males,
allowing for simultaneous in vivo recordings. Interestingly, the study
also shows that tarsal contact with a target female is enough to trigger
initial, but not late, stages of courtship. Overall, the paper takes
a significant step forward towards understanding the neural circuitry
of pheromone processing and pheromone guided behavior.

Papers of particular interest, published within the period of review,
have been highlighted as:

• of special interest

•• of outstanding interest



\newlist{biblist}{enumerate}{1}%
\setlist[biblist]{label={[\arabic*]}}%
\defbibenvironment{bibliography}
{\biblist{}
{\setlength{\leftmargin}{\bibhang}%
\setlength{\itemindent}{-\leftmargin}%
\setlength{\itemsep}{\bibitemsep}%
\setlength{\parsep}{\bibparsep}}}
{\endenumerate}
{\item}
\printbibliography


\section*{Figure Legends}


\subsection*{Table 1 \emph{Drosophila} pheromonal stimuli and their receptors}

\textcolor{black}{\footnotesize{}M, male, F, female, PAA, phenylacetic
acid, PA, phenylacetaldehyde, }{\footnotesize{}7-T, (z)-7-tricosene,
7-P, 7-pentacosene, 9-T, z-9-tricosene, 11-P, z-11-pentacosene, 7,11-HD,
(7Z,11Z)-heptacosadiene, 7,11-ND, (7Z,11Z)-nonacosadiene,}\textcolor{black}{\footnotesize{}
SP, sex peptide, }\textcolor{blue}{\emph{\footnotesize{}fru+}}\textcolor{black}{\footnotesize{},\textsuperscript{\textcolor{black}{\footnotesize{}1}}
unclear whether }\textcolor{black}{\emph{\footnotesize{}ppk29}}\textcolor{black}{\footnotesize{}
neurons co-express }\textcolor{black}{\emph{\footnotesize{}fru}}\textcolor{black}{\footnotesize{},
}\textsuperscript{2}\textcolor{black}{\footnotesize{}i.e. directly
acting on target tissue, \textsuperscript{\textcolor{black}{\footnotesize{}3}}
detected by }\textcolor{black}{\emph{\footnotesize{}ppk23+}}\textcolor{black}{\footnotesize{}
neurons \citep{Toda:2012ff}, \textsuperscript{\textcolor{black}{\footnotesize{}4}}cVA
acts as an aggregation pheromone only in conjunction with food odors
(see \citep{Bartelt:1985bh}), }\textsuperscript{5}\textcolor{black}{\footnotesize{}Gr32a
(but not Gr33a) required for inhibition of interspecies courtship
\citep{Fan:2013fv}.}{\footnotesize \par}


\subsection*{Figure 1 Pheromone perception in Drosophila courtship}

\textbf{\footnotesize{}(a)}{\footnotesize{} Proposed model for how
pheromonal cues guide male behavior at different spatial ranges. Note
that neuron names are color-coded to match the illustrations in }\textbf{\footnotesize{}(b)}{\footnotesize{}
and }\textbf{\footnotesize{}(c)}{\footnotesize{}.}{\footnotesize \par}

\noindent \textbf{\footnotesize{}(b)}{\footnotesize{} Hypothetical
wiring diagram from chemosensory inputs to descending motor output.
Darker arrows suggest strong experimental evidence for synaptic connections
and the direction of information flow, whereas lighter arrows suggest
probable connections based on neuroanatomical data, i.e. physical
overlap (see }\textbf{\footnotesize{}(c)}{\footnotesize{}). The putative
connection from mAL to pMP4/P1 may be inhibitory\citep{Fan:2013fv},
suggesting that the wiring diagram above is incomplete, as pMP4/P1
neurons must get excitatory input from somewhere else.}{\footnotesize \par}

\noindent \textbf{\footnotesize{}(c)}{\footnotesize{} 3D renderings
of neurons involved in pheromone processing (left), sensory integration
(middle), and motor control (right). Black arrowheads illustrate sensory
input (left) or motor output (right), respectively. }{\footnotesize \par}

\noindent {\footnotesize{}Fly cartoon images adapted with permission
from Sokolowski, Nature Reviews Genetics 2:879-890 (2001).}{\footnotesize \par}

\nocite{philipsborn2011,Kohatsu:2011aa}


\subsection*{Box 1 Open questions in \emph{Drosophila} pheromone processing}


\section*{Figures}

\begin{table}
\protect\caption{\textbf{\label{tab:pheromones-receptors}}}


\begin{tabular*}{0.85\paperwidth}{@{\extracolsep{\fill}}|>{\raggedright}p{0.07\paperwidth}|>{\centering}p{0.04\paperwidth}|>{\centering}p{0.06\paperwidth}|>{\centering}p{0.06\paperwidth}|>{\centering}p{0.03\paperwidth}|>{\centering}p{0.12\paperwidth}|>{\centering}p{0.2\paperwidth}|>{\centering}p{0.07\paperwidth}|}
\hline 
\textbf{Pheromone} & \textbf{Range} & \textbf{emitter -> detector} & \textbf{Receptor} & \textbf{PN} & \textbf{Behavior} & \textbf{Comments} & \textbf{References}\tabularnewline
\hline 
\hline 
\multirow{5}{0.07\paperwidth}{cVA} & \multirow{5}{0.04\paperwidth}{volatile \& contact} & M -> M & \textcolor{blue}{Or67d} & \textcolor{blue}{DA1} & aggression & suppressed by chronic cVA via Or65a & \citep{Wang:2010kl,Ejima:2007bs,Liu:2011dq}\tabularnewline
\cline{3-8} 
 &  & F -> M & \textcolor{blue}{Or67d} & \textcolor{blue}{DA1} & male repulsion & transferred to female by prior mating & \citep{Kurtovic:2007fu}\tabularnewline
\cline{3-8} 
 &  & M -> F & \textcolor{blue}{Or67d} & \textcolor{blue}{DA1} & female receptivity & suppressed by chronic cVA via Or65a & \citep{Kurtovic:2007fu,Lebreton:2014cr}\tabularnewline
\cline{3-8} 
 &  & M/F -> M/F & Or65a (?) & DL3 (?) & aggregation\textsuperscript{4} &  & \citep{Bartelt:1985bh}\tabularnewline
\cline{3-8} 
 &  & M -> M & \textcolor{blue}{\emph{ppk23}}\textcolor{blue}{/}\textcolor{blue}{\emph{ppk29}}\textsuperscript{1} & n/a & male-male repulsion & \emph{ppk} channels pheromone receptors ? & \citep{Thistle:2012fu}\tabularnewline
\hline 
\multirow{3}{0.07\paperwidth}{7-T} & \multirow{3}{0.04\paperwidth}{contact} & M -> F & Gr32a & n/a & female receptivity & increases female receptivity & \citep{Grillet:2006tg}\tabularnewline
\cline{3-8} 
 &  & M -> M & Gr32a (Gr66a) & n/a & male-male / interspecies repulsion &  & \citep{Fan:2013fv,Lacaille:2007ij}\tabularnewline
\cline{3-8} 
 &  & M -> M & \textcolor{blue}{\emph{ppk23}}\textcolor{blue}{/}\textcolor{blue}{\emph{ppk29}}\textsuperscript{1} & n/a & male-male repulsion & \emph{ppk} channels pheromone receptors ? & \citep{Toda:2012ff,Starostina:2012mi,Liu:2012ys,Thistle:2012fu}\tabularnewline
\hline 
7-P & contact & M -> F & ? & n/a & female receptivity (?) & male-enriched & \citep{Ferveur:2005ly,Yew:2009fv}\tabularnewline
\hline 
7,11-HD (7,11-ND)\textsuperscript{3} & contact & F -> M & \textcolor{blue}{\emph{ppk23}}\textcolor{blue}{/}\textcolor{blue}{\emph{ppk29}}\textcolor{blue}{/}

\textcolor{blue}{\emph{ppk25}}\textcolor{blue}{/}\textcolor{blue}{\emph{nope}}\textsuperscript{1} & n/a & male courtship & \emph{ppk} channels pheromone receptors ? & \citep{Toda:2012ff,Starostina:2012mi,Liu:2012ys,Thistle:2012fu}\tabularnewline
\hline 
9-P & contact & F -> M & ? & n/a & male copulation & role in courtship conditioning & \citep{Siwicki:2005kl}\tabularnewline
\hline 
\hline 
CH503 & contact (?) & F -> M & ? & ? & male repulsion & transferred to female by prior mating & \citep{Yew:2009fv}\tabularnewline
\hline 
\hline 
9-T, 11-P & contact & M -> M & Gr32a / Gr33a\textsuperscript{5} & n/a & male-male / inter- species repulsion & no cVA-mediated aggression in Gr32a -/- males; contact \emph{fru+}
neurons in GNG & \citep{Svetec:2005kl,Lacaille:2007ij,Miyamoto:2008kl,Moon:2009qa,Koganezawa:2010tg,Fan:2013fv}\tabularnewline
\hline 
\hline 
PAA, PA & volatile & food -> M & \textcolor{blue}{Ir84a} & \textcolor{blue}{VL2a} & male courtship & kairomone aphrodisiac & \citep{Grosjean:2011fu}\tabularnewline
\hline 
\hline 
SP & contact & M -> F & \textcolor{blue}{SPR} & n/a & postmating response & allohormone pheromone\textsuperscript{2}, transferred with seminal
fluid & \citep{Yapici:2008ve}\tabularnewline
\hline 
\hline 
M/F extract & contact & ? & \textcolor{blue}{Or47b} & \textcolor{blue}{VA1v} & ? & active compound(s) unidentified & \citep{Goes-van-Naters:2007dq}\tabularnewline
\hline 
\hline 
M/F extract & contact & ? & \textcolor{blue}{Or88a} & \textcolor{blue}{VA1d} & ? & active compound(s) unidentified & \citep{Goes-van-Naters:2007dq}\tabularnewline
\hline 
? & contact & F -> M & Gr68a & n/a & male courtship & stimulate male courtship, \emph{doublesex}-dependent & \citep{Bray:2003fk}\tabularnewline
\hline 
? & contact & F -> M & Gr39a & n/a & male courtship & role in sustaining male courtship behavior & \citep{Watanabe:2011hc}\tabularnewline
\hline 
? & contact & F -> M & IR52c / IR52d & n/a & male courtship & presumably contact \emph{fru+} neurons in prothoracic ganglia & \citep{Koh:2014cr}\tabularnewline
\hline 
\end{tabular*}
\end{table}


\newpage{}

\begin{figure}
\protect\caption{\textbf{\label{fig:courtship}}}


%\includegraphics[width=0.8\paperwidth]{figures/FlyCourtshipModel}
\end{figure}


\newpage{}

\begin{figure}
\protect\caption{\textbf{\label{fig:Questions}}}

\begin{itemize}
\item What similarities and differences are there in processing non-pheromone
odors, especially those that have strong intrinsic valence?
\item How can we explain the different roles of cVA (i.e. courtship vs aggression
vs aggregation)?
\item What are the ligands for OR47b?
\item What is the role of PAA/PA detection by IR84a in females?\textcolor{blue}{{}
\citep{Grosjean:2011fu}}
\item What is the identity and tuning of second-order gustatory neurons? 
\item What is the role of \emph{ppk23+, ppk25+, ppk29+} neurons in females?
\item Where and how (i.e. sublinear/linear/supralinear) does integration
of pheromone signals take place (e.g. see \citep{Gruntman:2013vn})?
\item Are cVA signals from tarsal \emph{ppk+} neurons and higher olfactory
neurons integrated?
\item Does \emph{fruitless} specify a dedicated circuit for pheromone perception?
\item Are there other labeled lines? If yes, how many are there and which
genes specify them developmentally?
\item How plastic is pheromone processing and what mechanisms exist (e.g.
receptor desensitization, plasticity of central circuits)
\item Where and how do circuits for innate (or: pheromone-driven) and learned
behaviors interact? (e.g. see \citep{Keleman:2012dz})
\item How do courtship promoting and inhibiting pheromonal cues interact
to guide different stages of courtship behavior?\end{itemize}
\end{figure}

\end{document}
